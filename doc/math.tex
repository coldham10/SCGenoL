\documentclass{article}
\usepackage[margin=1in]{geometry}
\usepackage{amsmath}
\usepackage{amssymb}

\begin{document}

\section{Genotype Likelihoods from Reads}
\begin{align*}
L(g=1) &= p(d\,|\,g=1)\\
&=p(d, g=ra \,|\, g=1) + p(d, g=rb \,|\, g=1) + p(d, g=rc \,|\, g=1)\\
&= \sum_a p(d \,|\, g=ra) \times p(g=ra \,|\, g=1)
\end{align*}
Here we do not favour any heterozygous genotype, and all have likelihood $1/3$. This may be changed to reflect empirical or dbSNP data
\begin{align*}
L(g=1) &= 1/3 \sum_a p(d \,|\, g=ra)\\
&= 1/3 \sum_a p(d \,|\, g=ra, \text{ado}) \times p(\text{ado}) + p(d \,|\, g=ra, \text{no ado})(1-p(\text{ado}))\\
p(d|g=ra, ado) &= p(d|g=ra, drop r) * p(drop r) + p(d|g=ra, drop a) * p(drop a)
\end{align*}
Here we assume either allele is equally likely to be dropped in an ado event and $p(\text{drop r}) = p(\text{drop a}) = 0.5$. This is unlikely to change.
\begin{align*}
p(d \,|\, g=ra, \text{ado}, \text{drop a}) &= \prod_i p(d_i \,|\, g=rr)\\
p(d \,|\, g=ra, \text{ado}, \text{drop r}) &= \prod_i p(d_i \,|\, g=aa)
\end{align*}

\section{Cell-locus Posterior probabilities}
Using Bayes' rule:
\begin{align*}
p(g=k\,|\, d) &= \frac{p(d\,|\,g=k)\,p(g=k)}{p(d)}\\
&= \frac{p(d\,|\,g=k)\,\mu^k}{p(d)}\\
\end{align*}
where k is the mutation rate, a learnable parameter. Note: this parameter may be overestimated if the algorithm finds more mutations, increases the rate prior, and so finds more mutations. There may be no reason for this to converge.

\end{document}
