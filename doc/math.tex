\documentclass{article}
\usepackage[margin=1in]{geometry}
\usepackage{amsmath}
\usepackage{amssymb}

\begin{document}

\section{Genotype Likelihoods from Reads}
\begin{align*}
L(g=1) &= p(d\,|\,g=1)\\
&=p(d, g=ra \,|\, g=1) + p(d, g=rb \,|\, g=1) + p(d, g=rc \,|\, g=1)\\
&= \sum_a p(d \,|\, g=ra) \times p(g=ra \,|\, g=1)
\end{align*}
Here we do not favour any heterozygous genotype, and all have likelihood $1/3$. This may be changed to reflect empirical or dbSNP data
\begin{align*}
L(g=1) &= 1/3 \sum_a p(d \,|\, g=ra)\\
&= 1/3 \sum_a p(d \,|\, g=ra, \text{ado}) \times p(\text{ado}) + p(d \,|\, g=ra, \text{no ado})(1-p(\text{ado}))\\
p(d|g=ra, ado) &= p(d|g=ra, drop r) * p(drop r) + p(d|g=ra, drop a) * p(drop a)
\end{align*}
Here we assume either allele is equally likely to be dropped in an ado event and $p(\text{drop r}) = p(\text{drop a}) = 0.5$. This is unlikely to change.
\begin{align*}
p(d \,|\, g=ra, \text{ado}, \text{drop a}) &= \prod_i p(d_i \,|\, g=rr)\\
p(d \,|\, g=ra, \text{ado}, \text{drop r}) &= \prod_i p(d_i \,|\, g=aa)
\end{align*}

\section{Cell-locus Posterior probabilities}
Using Bayes' rule:
\begin{align*}
p(g=k\,|\, d) &= \frac{p(d\,|\,g=k)\,p(g=k)}{p(d)}\\
&= \frac{p(d\,|\,g=k)\,\binom{2}{k}f_1^k\,(1-f_1)^{2-k}}{p(d)}\\
\end{align*}
where $f_1$ is the alternate allele frequency at that site. This implies HWE, which may or may not be a valid assumption. Since $p(d)$ is the same for all values of $k$ at a cell-locus, we do not need to find it and can simply normalise.
\subsection{Priors}
Above is the current implementation. The alernate allele frequency may be estimated by EM at each site. Other options exist such as:
\subsubsection*{Marginalizing by site allele count}
As done by Zafar et al., first probabilities for the number of alternate alleles $l$ at the site are calculated using dynamic programing.\\
\begin{align*}
p(g=k) &= \sum\limits_{l'=k}^{2m-2+k} p(g=k \,|\,l=l') \,p(l=l')\\
&=
\end{align*}
\subsection*{Site frequency spectrum}
\subsection*{Phylogeny aware prior}
As done by Singer et al.. Similar to Zafar et al. except we cosider the number of affected cells, $K$ rather than the number of affected alleles. 
This prior includes the probability of a given site containing a mutation ($\lambda$) as well as a distribution of the number of cells affected. For $P(K=0)$ is simply $1-\lambda$. for $K\neq 0 $:
\[p(K=k) = \lambda\frac{\binom{m}{k}^2}{(2k-1)\binom{2m}{2k}}\]

\section{Probabilistic Hamming distance}
The Hamming distance between two sequences $s,p$ both length $n$ is given by \[\sum\limits_i^n (1-\delta_{s_ip_i})\] where $\delta_{ab}$ is the Kronecker delta.
Since we have a probabilistic tree, we use a similar metric but weighted by the posterior probabilities of the genotypes at each locus:
\[\sum\limits_i^n \sum\limits_{(a,b)\in\{0,1\}\times\{0,1\}}(1-\delta_{ab})p(s_ip_i = ab)\]
Since $(1-\delta_{ab})$ vanishes when $s_i = p_i$, the distance reduces to
\[\sum\limits_i^n p(s_i = 0\,,\,p_i=1)+p(s_i=1\,,\,p_i=0)\]
If we assume independce (TODO) we have
\[D_{s,p} = \sum\limits_i^n p(s_i = 0)p(p_i=1)+p(s_i=1)p(p_i=0)\]

\end{document}
